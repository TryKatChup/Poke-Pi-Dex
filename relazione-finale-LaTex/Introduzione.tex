\chapter{Introduzione}
\label{ch:Introduction}
\section{Pokémon}

Breve introduzione storica su cosa sono e perché sono famosi.~\cite{bijeeta}
Ovviamente parlare anche dell'esistenza del videogioco, del gioco di carte e dell'anime.

\section{Pokédex}
Raccontare in breve cosa fa un pokédex e perché risulta importante.
Spiegare che è una enciclopedia in grado di riconoscere i Pokémon che si incontra. Parlare del fatto che ora siamo quasi a 1000 Pokémon.

\section{Scopo del progetto}

L'obiettivo del nostro progetto è quello di riconoscere in modo automatico i Pokémon presenti sotto forma di peluche, come carte da gioco, fanart, immagini del gioco tramite l'utilizzo di reti neurali convoluzionali (CNN). A tal proposito è stato realizzato un dispositivo fisico simile
a un Pokédex su cui verrà effettuato il deployment della mia rete neurale.

La relazione è suddivisa nei seguenti capitoli:
\begin{itemize}
  \item nel capitolo 2 è descritto il modello della rete, il preprocessing dei dati, le diverse scelte effettuate in fase di training, e i risultati ottenuti;
  \item nel capitolo 3 è presente l'implementazione della applicazione sul dispositivo fisico e .......... TODO;
  \item TODO
\end{itemize}

Tutto il codice del progetto può essere recuperato al seguente link: 
\\
https://github.com/TryKatChup/pokemon-cv-revival

Magari mettere i link per ciascun pezzo di codice o paragrafo.
\newline
L'obiettivo del nostro progetto è quello di riconoscere in modo automatico i Pokémon presenti sotto forma di peluche, come carte da gioco, fanart, immagini del gioco tramite l'utilizzo di reti neurali convoluzionali (CNN). A tal proposito è stato realizzato un dispositivo fisico simile
a un Pokédex su cui verrà effettuato il deployment della mia rete neurale.

La relazione è suddivisa nei seguenti capitoli:
\begin{itemize}
  \item nel capitolo 2 è descritto il modello della rete, il preprocessing dei dati, le diverse scelte effettuate in fase di training, e i risultati ottenuti;
  \item nel capitolo 3 è presente l'implementazione della applicazione sul dispositivo fisico e .......... TODO;
  \item TODO
\end{itemize}

Tutto il codice del progetto può essere recuperato al seguente link: 
\\
https://github.com/TryKatChup/pokemon-cv-revival

Magari mettere i link per ciascun pezzo di codice o paragrafo.
\\
L'obiettivo del nostro progetto è quello di riconoscere in modo automatico i Pokémon presenti sotto forma di peluche, come carte da gioco, fanart, immagini del gioco tramite l'utilizzo di reti neurali convoluzionali (CNN). A tal proposito è stato realizzato un dispositivo fisico simile
a un Pokédex su cui verrà effettuato il deployment della mia rete neurale.

La relazione è suddivisa nei seguenti capitoli:
\begin{itemize}
  \item nel capitolo 2 è descritto il modello della rete, il preprocessing dei dati, le diverse scelte effettuate in fase di training, e i risultati ottenuti;
  \item nel capitolo 3 è presente l'implementazione della applicazione sul dispositivo fisico e .......... TODO;
  \item TODO
\end{itemize}

Tutto il codice del progetto può essere recuperato al seguente link: 
\\
https://github.com/TryKatChup/pokemon-cv-revival

Magari mettere i link per ciascun pezzo di codice o paragrafo.
\\
L'obiettivo del nostro progetto è quello di riconoscere in modo automatico i Pokémon presenti sotto forma di peluche, come carte da gioco, fanart, immagini del gioco tramite l'utilizzo di reti neurali convoluzionali (CNN). A tal proposito è stato realizzato un dispositivo fisico simile
a un Pokédex su cui verrà effettuato il deployment della mia rete neurale.

La relazione è suddivisa nei seguenti capitoli:
\begin{itemize}
  \item nel capitolo 2 è descritto il modello della rete, il preprocessing dei dati, le diverse scelte effettuate in fase di training, e i risultati ottenuti;
  \item nel capitolo 3 è presente l'implementazione della applicazione sul dispositivo fisico e .......... TODO;
  \item TODO
\end{itemize}

Tutto il codice del progetto può essere recuperato al seguente link: 
\\
https://github.com/TryKatChup/pokemon-cv-revival

Magari mettere i link per ciascun pezzo di codice o paragrafo.
\\
L'obiettivo del nostro progetto è quello di riconoscere in modo automatico i Pokémon presenti sotto forma di peluche, come carte da gioco, fanart, immagini del gioco tramite l'utilizzo di reti neurali convoluzionali (CNN). A tal proposito è stato realizzato un dispositivo fisico simile
a un Pokédex su cui verrà effettuato il deployment della mia rete neurale.

La relazione è suddivisa nei seguenti capitoli:
\begin{itemize}
  \item nel capitolo 2 è descritto il modello della rete, il preprocessing dei dati, le diverse scelte effettuate in fase di training, e i risultati ottenuti;
  \item nel capitolo 3 è presente l'implementazione della applicazione sul dispositivo fisico e .......... TODO;
  \item TODO
\end{itemize}

Tutto il codice del progetto può essere recuperato al seguente link: 
\\
https://github.com/TryKatChup/pokemon-cv-revival

Magari mettere i link per ciascun pezzo di codice o paragrafo.
